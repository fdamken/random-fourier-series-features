(Deep) \acp{NN} are extremely powerful and expressive machine learning models dominating the current landscape of artificial intelligence\cite{krizhevskyImageNetClassificationDeep2012}.
While they have great predictive power, \acp{NN} usually lack uncertainty estimation and a Bayesian treatment.
In Bayesian machine learning, we not only seek models that predict some value, but that also gauge their uncertainty about the prediction.
While frequentistic models are capable of estimating aleatoric (i.e., noise-induced) uncertainty, Bayesian models include epistemic uncertainty that quantifies the model's trust "in itself."
Quantification of the uncertainty is useful in a variety of domains such as \ac{MPC}\cite{hewingLearningBasedModelPredictive2020,bradfordStochasticDatadrivenModel2020}.

\begin{figure}
	\centering
	\includegraphics[width=\linewidth]{graphics/hypercube.tikz}
	\caption{
		Connections between various (Bayesian) regression models.
		The three axis distinct them between using kernels vs. features (blue), relying on manually designing vs. learning (yellow), and stationary vs. non-stationary behavior (green).
		\acsp{RFSF} (our method) are not clearly stationary or non-stationary and theoretical analysis is up to future research.
	}
	\label{fig:hypercube}
\end{figure}

A proposed class of extensions for \acp{NN} are \acp{BNN}\cite{mackayPracticalBayesianFramework1992,nealBayesianLearningNeural2012} combining the expressiveness of \acp{NN} with uncertainty quantification.
However, exact inference is intractable, so these methods have to resort to approximate inference approaches\cite{nealBayesianLearningNeural2012,hernandez-lobatoProbabilisticBackpropagationScalable2015,denkerTransformingNeuralNetOutput1990,galDropoutBayesianApproximation2016,lakshminarayananSimpleScalablePredictive2017,blundellWeightUncertaintyNeural2015}.
Also, \acp{BNN} suffer from various drawbacks such as expensive and complicated training, inaccurate posteriors, and unreliable uncertainty quantification\cite{foongExpressivenessApproximateInference2020,foongInBetweenUncertaintyBayesian2019,osbandRandomizedPriorFunctions2018,ovadiaCanYouTrust2019,wenzelHowGoodBayes2020,yaoQualityUncertaintyQuantification2019}.
Hence, they are not suitable for application in high-stakes domains such as medical diagnosis where accurate uncertainty quantification is necessary\cite{watsonLatentDerivativeBayesian2021}.

A well-known alternative approach for Bayesian machine learning are \acp{GP}.
\aclp{GP} allow exact inference leveraging linear regression.
Using a kernel for lifting the inputs into a high- and possible infinite-dimensional feature space allows for great flexibility.
Despite the great uncertainty estimation of \acp{GP}, their raw prediction power is limited and highly dependent on the kernel.
This dependence is reflected in the amount of kernels that have been studied throughout the years\cite[ch.\,2]{duvenaudAutomaticModelConstruction2014}.
Motivated by this limitation, methods for (deep) kernel learning (\acsu{DKL}) have been developed\cite{wilsonDeepKernelLearning2016,calandraManifoldGaussianProcesses2016,jacotNeuralTangentKernel2020}.
However, exact inference with kernels is a tedious task requiring inversion of an \(N \times N\)-matrix (the Gram matrix) where \(N\) is the number of data points.
The computational complexity of this is cubic, prohibiting online use of \acp{GP} for large data sets\cite{rahimiRandomFeaturesLargeScale2007}.
Approaches for reducing the computational complexity have been proposed such as inducing points\cite{snelsonSparseGaussianProcesses2005} or the Nyström method\cite{nystromUberPraktischeAuflosung1930,sunReviewNystromMethods2015}.

Alternatively, one can resort to Bayesian linear regression with features.
To mimic \ac{GP} regression, these features are chosen to approximate a kernel, e.g., the \ac{SE} kernel.
A well-known choice are \acp{RFF} that can approximate arbitrary kernels and are usually configured for the \ac{SE} kernel as this is possible in closed form\cite{rahimiRandomFeaturesLargeScale2007}.
However, the \ac{SE} kernel usually produces results that are too smooth\cite{steinInterpolationSpatialData1999} and is, by design, not able to capture functions with non-stationary length-scale.

\emph{Contribution:} We propose an extension of \acp{RFF}, \acp{RFSF}, that build a bridge between \acp{RFF}, \ac{DKL}, \acp{BNN}, and  by reducing the computational complexity of \ac{GP} regression due to working with features, enrich the capacity of \acp{GP} by adding more parameters and reducing the need of (manually) designed kernels, and using classical training methods known from \acp{NN} for optimizing the hyper-parameters.
\Cref{fig:hypercube} illustrates these connections on three axis: kernels vs. features, designing vs. learning, and stationary vs. non-stationary length-scales.
As theoretical analysis of \acp{RFSF} are up to future work, it is open whether they belong to the stationary or non-stationary category.
Nonetheless, out empirical results suggest that they are non-stationary.

We will now highlight connections to related work (\cref{sec:relatedWork}).
Subsequently, we will lay out some preliminaries (\cref{sec:preliminaries}), present the methodology of \acp{RFSF} in \cref{sec:methods}, and finally empirically evaluate and summarize our findings as well as provide direction for future research (\cref{sec:eval,sec:conclusion}).
