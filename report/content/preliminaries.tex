\subsection{Fourier Series}
	\emph{Fourier series} are a principled way to represent any periodic function (satisfying the Dirichlet conditions\cite{oppenheimSignalsSystems1997}) as a linear combination of sine and cosine waves.
	The resulting approximation $\hat{f}_K$ of the function $f$ can be represented in three different fashions: in sine-cosine form, amplitude-phase form, and exponential form.
	In the sine-cosine form, the series is represented by the sum of separate sine and cosine waves with separate amplitudes,
	\begin{equation}
		\hat{f}_K(x) = \frac{a_0}{2} + \sum_{k = 1}^{K} a_k \cos(\pi k x / \tilde{T}) + b_k \sin(\pi k x / \tilde{T}),
		\label{eq:sineCosineFourierSeries}
	\end{equation}
	where the Fourier coefficients $a_k$ and $b_k$ are given by
	\begin{align}
		a_k &= \frac{1}{\tilde{T}} \int_{x_0}^{x_0 + 2\tilde{T}} f(x) \cos(\pi k x / \tilde{T}) \dd{x} \\
		b_k &= \frac{1}{\tilde{T}} \int_{x_0}^{x_0 + 2\tilde{T}} f(x) \sin(\pi k x / \tilde{T}) \dd{x}.
	\end{align}
	Here, $\tilde{T}$ is half the period of $f$ (for instance, $f(x) = \sin(x)$ has the half-period $\tilde{T} = \pi$).
	In amplitude-phase form,
	\begin{equation}
		\hat{f}_K(x) = \frac{A_0}{2} + \sum_{k = 1}^{K} A_k \cos(2 \pi k x / \tilde{T} - \varphi_k),
	\end{equation}
	the series is represented by individual cosine waves with amplitudes and phases.
	These can be computed from the Fourier coefficients as follows:
	\begin{align}
		A_k &= \sqrt{a_k^2 + b_k^2} &
		\varphi_k &= \mathrm{arctan2}(b_k, a_k).
	\end{align}
	For the following, we stick to the former version (sine-cosine) for practical reasons:
	the amplitude-phase formulation introduces some ambiguity as $\varphi \equiv \varphi + 2\pi$.
	When optimizing numerically, this can cause problems due to either having ambiguous optima or having to include constraints.
% end

\subsection{Gaussian Process Regression}
%	In supervised learning, we are concerned with modeling a function $f : \mathcal{X} \to \mathcal{Y}$ from input data $\mathcal{X}$ to labels $\mathcal{Y}$ using an (approximate) model $\hat{f} : \mathcal{X} \to \mathcal{Y}$.
%	The model is trained on a labeled dataset $\mathcal{D} = \{ (\vec{x}_i, y_i) \}_{i = 1, \dots, N} \subseteq \mathcal{X} \times \mathcal{Y}$.
%	For regression, the target labels $\mathcal{Y}$ are continuous: $\mathcal{Y} \subseteq \R$.
%	Various methods exist for tackling this problem, e.g., linear regression, \acp{SVM}, and \acp{NN}.
%	Bayesian methods take this idea one step further and not only model the mapping from inputs to targets but also quantify their epistemic uncertainty.
%	This idea can be incorporated into linear regression, for example, by placing a prior on the feature weights and computing the posterior of new data given the training data.
%	Gauging the (epistemic) uncertainty is helpful in various domains such as robust \ac{MPC}.
%	Incorporating uncertainty allows the controller to vary its trust in the underlying model, avoiding unknown and potentially unsafe terrain.

	\ac{GP} regression can be viewed in two fashions:
	firstly, as an extension of linear regression to an infinite number of features.
	This involves computing the limit $n \to \infty$ where $n$ is the number of features.
	Secondly, as an infinite Gaussian distribution over a function space where every finite subset of the random variables is jointly Gaussian.
	However, both definitions (or views) yield the exact same result.
	For the rest of this report we will stick to the former definition.

	We closely follow\cite{rasmussenGaussianProcessesMachine2006} in terms of notation, refer to them for a more thorough treatment of \ac{GP} regression and its derivation, and will give a brief overview over the used notation.
	Let \(\vec{x}_\ast\) and \(y_\ast\) be a test input and target\footnote{When predicting, the target is not known and the \ac{GP} defines a Gaussian distribution over it.}, respectively, then the \ac{GP} regression posterior is a Gaussian distribution with the following mean and variance:
	\begin{equation}
		\begin{aligned}
			\E[y_\ast] &= \vec{k}_\ast^\transposed \mat{K}^{-1} \vec{y} &
			\V[y_\ast] &= k_\ast - \vec{k}_\ast^\transposed \mat{K}^{-1} \vec{k}_\ast
		\end{aligned}
		\label{eq:gp}
	\end{equation}
	Here, \( \mat{K} \) denotes the Gram matrix on the training inputs, \(\vec{k}_\ast\) denotes the evaluation of the kernel between the test input and train inputs organized into a column vector, and \(k_\ast\) is the kernel evaluation of the test input \(\vec{x}_\ast\); train targets are organized into \(\vec{y}\).

	An exemplary kernel is the \ac{SE} kernel \( k_\mathrm{SE}(\vec{x} - \vec{y}) = \exp{ -\lVert \vec{x} - \vec{y} \rVert_2^2 / (2 \ell^2) } \) with the length-scale\footnote{The length-scale determines the "roughness" of the function samples\cite[p.\,15]{rasmussenGaussianProcessesMachine2006}.} \(\ell^2\).
	However, it has been shown that this kernel often exhibits extremely smooth functions (as it is infinitely differentiable) often unrealistic for real-world data\cite{steinInterpolationSpatialData1999}.
	% TODO: When at the ULB before submitting this thing: search specific page in Stein's book.
	Despite its drawbacks, the \ac{SE} kernel remains one of the most widely used kernels due to its simplicity and applicability\cite[p.\,83]{rasmussenGaussianProcessesMachine2006}.

	While employing the "kernel trick" allows great flexibility with (potentially) infinite-dimensional feature spaces, it introduces major computational challenges due to the inversion of the Gram matrix in \cref{eq:gp}.
	That is, the inversion has time complexity \( \mathcal{O}(N^3) \), where \(N\) is the number of data points in the training data set.
%	One approach for tackling this problem are \acl{RFF} that resort to Bayesian regression by choosing appropriate features that approximate a desired kernel.
	One approach for tackling this problem are \acl{RFF}.
% end
